% $Header$

\documentclass{beamer}

% This file is a solution template for:

% - Giving a talk on some subject.
% - The talk is between 15min and 45min long.
% - Style is ornate.



% Copyright 2004 by Till Tantau <tantau@users.sourceforge.net>.
%
% In principle, this file can be redistributed and/or modified under
% the terms of the GNU Public License, version 2.
%
% However, this file is supposed to be a template to be modified
% for your own needs. For this reason, if you use this file as a
% template and not specifically distribute it as part of a another
% package/program, I grant the extra permission to freely copy and
% modify this file as you see fit and even to delete this copyright
% notice. 


\mode<presentation>
{
  \usetheme{Warsaw}
%  \useinnertheme{circles}
  \setbeamercovered{transparent}
  \setbeamertemplate{navigation symbols}{}
  \setbeamertemplate{page number in head/foot}[totalframenumber]
}

\usepackage[utf8]{inputenc}
% or whatever

\usepackage[english, russian]{babel}
% or whatever

\usepackage{tikz}

% Or whatever. Note that the encoding and the font should match. If T1
% does not look nice, try deleting the line with the fontenc.


\title % (optional, use only with long paper titles)
{Алгоритмы классификации данных}

\subtitle
{Анализ алгоритмов} % (optional)

\author[Пиминов А.Э.] % (optional, use only with lots of authors)
{Пиминов Артур Эдуардович}
% - Use the \inst{?} command only if the authors have different
%   affiliation.

\institute[НГУ] % (optional, but mostly needed)
{
  Группа 23224\\
  Факультет информационных технологий\\
  Новосибирский государственный университет
}
\date[09.11.2023] % (optional)
{9 ноября 2023 г.}

\subject{Talks}
% This is only inserted into the PDF information catalog. Can be left
% out. 

% If you have a file called "university-logo-filename.xxx", where xxx
% is a graphic format that can be processed by latex or pdflatex,
% resp., then you can add a logo as follows:

\pgfdeclareimage[height=1cm]{university-logo}{university-logo-filename}
\logo{\pgfuseimage{university-logo}}


% Delete this, if you do not want the table of contents to pop up at
% the beginning of each subsection:
\AtBeginSubsection[]
{
  \begin{frame}<beamer>{Cодержание}
    \tableofcontents[currentsection,currentsubsection]
  \end{frame}
}


% If you wish to uncover everything in a step-wise fashion, uncomment
% the following command: 

%\beamerdefaultoverlayspecification{<+->}

\begin{document}

\begin{frame}
  \titlepage
\end{frame}

\begin{frame}{Содержание}
  \tableofcontents
  % You might wish to add the option [pausesections]
\end{frame}


% Since this a solution template for a generic talk, very little can
% be said about how it should be structured. However, the talk length
% of between 15min and 45min and the theme suggest that you stick to
% the following rules:  

% - Exactly two or three sections (other than the summary).
% - At *most* three subsections per section.
% - Talk about 30s to 2min per frame. So there should be between about
%   15 and 30 frames, all told.

\section{Введение}

\subsection{Понятие задачи классификации данных}

\begin{frame}{Определение}
  \begin{definition} 
   Классификация данных — разбиение множества объектов или наблюдений на априорно заданные группы, называемые классами, внутри каждой из которых они предполагаются похожими друг на друга, имеющими примерно одинаковые свойства и признаки. 
  \end{definition}
\end{frame}

\begin{frame}{Make Titles Informative.}

  You can create overlays\dots
  \begin{itemize}
    \item using the \texttt{pause} command:
      \begin{itemize}
        \item
          First item.
          \pause
        \item    
          Second item.
      \end{itemize}
    \item
      using overlay specifications:
      \begin{itemize}
        \item<3->
          First item.
        \item<4->
          Second item.
      \end{itemize}
    \item
      using the general \texttt{uncover} command:
      \begin{itemize}
        \uncover<5->{\item
        First item.}
        \uncover<6->{\item
        Second item.}
    \end{itemize}
\end{itemize}
\end{frame}


\subsection{Примеры}
\begin{centering}
  \begin{frame}{Классический пример кластеризации}
    \begin{tikzpicture}
      \draw (5,3) ellipse (1 and 2) node[above=60]{Группа 1};
      \foreach \s in {1,...,5}
      {
        \foreach \t in {1,...,5}
        {
          \filldraw[red] (4.4 + 1.3 * rnd ,2.1 + 2 * rnd) circle (0.1);
        }
      }
      \node[circle,fill=black, inner sep=0pt, minimum size=5] (e1) at (6,5) {};
      \node[circle,fill=black, inner sep=0pt, minimum size=5] (e2) at (7,5) {};
      \node[circle,fill=black, inner sep=0pt, minimum size=5] (e3) at (9,4) {};
      \node[draw] (A) at (7,3) {Выбросы};
      \draw[->] (A) -- (e1);
      \draw[->] (A) -- (e2);
      \draw[->] (A) -- (e3);
      \draw (10,3) ellipse (1 and 2) node[above=60]{Группа 2};
      \foreach \s in {1,...,5}
      {
        \foreach \t in {1,...,5}
        {
          \filldraw[green] (9.4 + 1.3 * rnd ,2.1 + 2 * rnd) circle (0.1);
        }
      }
    \end{tikzpicture}
  \end{frame}
\end{centering}


\begin{frame}{Make Titles Informative.}
\end{frame}


\section{Алгоритмы}
\subsection{Деревья решений}

\begin{frame}{Make Titles Informative.}
\end{frame}

\section*{Заключение}

\begin{frame}{Summary}

  % Keep the summary *very short*.
  \begin{itemize}
    \item
      The \alert{first main message} of your talk in one or two lines.
    \item
      The \alert{second main message} of your talk in one or two lines.
    \item
      Perhaps a \alert{third message}, but not more than that.
  \end{itemize}

  % The following outlook is optional.
  \vskip0pt plus.5fill
  \begin{itemize}
    \item
      Outlook
      \begin{itemize}
        \item
          Something you haven't solved.
        \item
          Something else you haven't solved.
      \end{itemize}
  \end{itemize}
\end{frame}

\begin{frame}
  \titlepage
\end{frame}

\end{document}



