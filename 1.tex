\PassOptionsToPackage{table,xcdraw}{xcolor}

\documentclass{beamer}

% This file is a solution template for:

% - Giving a talk on some subject.
% - The talk is between 15min and 45min long.
% - Style is ornate.

\mode<presentation>
{
  \usetheme{Warsaw}
  \setbeamercovered{transparent}
  \setbeamertemplate{navigation symbols}{}
  \setbeamertemplate{page number in head/foot}[totalframenumber]
  \setbeamertemplate{caption}[numbered]
  \setbeamertemplate{items}[circle]
  \useinnertheme{circles}
 }


\usepackage[utf8]{inputenc}
% or whatever

\usepackage[english, russian]{babel}
% or whatever

\usepackage{tikz}
\tikzstyle{process} = [rectangle, draw, text centered, minimum height=2em, align=center, fill=blue!20]
\tikzstyle{decision} = [circle, draw, text centered, minimum height=2em, align=center, fill=blue!20]
\tikzstyle{connector} = [draw, ->, >=triangle 45]
\tikzstyle{db} = [cylinder, 
        shape border rotate=90, 
        draw,
        minimum height=2cm,
        minimum width=2cm,
        shape aspect=.25, fill=red!20]
\tikzstyle{2connector} = [draw, <->, >=triangle 45]
\tikzstyle{terminator} = [rectangle, draw, text centered, rounded corners, minimum height=2em, align=center, fill=green!20]
\usetikzlibrary{shapes,shapes.geometric, arrows, arrows.meta}

\usepackage{caption}
\usepackage{subcaption}

\usepackage[citestyle=numeric,style=verbose,backend=biber]{biblatex}
\addbibresource{ref.bib}


\chardef\_=`_

\usepackage{array}

\title % (optional, use only with long paper titles)
{Алгоритмы классификации данных}

\subtitle
{Анализ алгоритмов} % (optional)

\author[Пиминов А.Э.] % (optional, use only with lots of authors)
{Пиминов Артур Эдуардович}
% - Use the \inst{?} command only if the authors have different
%   affiliation.

\institute[НГУ] % (optional, but mostly needed)
{
  Группа 23224\\
  Факультет информационных технологий\\
  Новосибирский государственный университет
}
\date[\today] % (optional)
{\today}

\subject{Talks}
% This is only inserted into the PDF information catalog. Can be left
% out. 

% If you have a file called "university-logo-filename.xxx", where xxx
% is a graphic format that can be processed by latex or pdflatex,
% resp., then you can add a logo as follows:

\pgfdeclareimage[height=1cm]{university-logo}{university-logo-filename}
\logo{\pgfuseimage{university-logo}}


% Delete this, if you do not want the table of contents to pop up at
% the beginning of each subsection:

\AtBeginSection[]
{
  \begin{frame}<beamer>{Cодержание}
    \tableofcontents[currentsection]
  \end{frame}
}


\AtBeginSubsection[]
{
  \begin{frame}<beamer>{Cодержание}
    \tableofcontents[currentsection,currentsubsection]
  \end{frame}
}


% If you wish to uncover everything in a step-wise fashion, uncomment
% the following command: 

\begin{document}

\begin{frame}
  \titlepage
\end{frame}

\begin{frame}{Содержание}
  \tableofcontents
  % You might wish to add the option [pausesections]
\end{frame}


% Since this a solution template for a generic talk, very little can
% be said about how it should be structured. However, the talk length
% of between 15min and 45min and the theme suggest that you stick to
% the following rules:  
%   15 and 30 frames, all told.

\section{Введение}

\subsection{Понятие задачи классификации данных}

\begin{frame}{Определение задачи классификации}
  \begin{block}{Определение} 
    \textbf{Классификация} — это разновидность анализа данных, в процессе которого извлекаются модели, описывающие важные классы данных\footnotemark. 
  \end{block}\footcitetext{Jiawei2012}
  \vfill
\end{frame}


\begin{frame}{Определение задачи классификации}
\begin{figure}
    \begin{tikzpicture}
      \node [process] at (0,0) (first) {\textbf{Классификаторы}};
      \node [terminator] at (0,-2) (second) {\textbf{Категориальные метки класса}};
      \node[draw=none] at (2, -1) (action) {Предсказывают};
      \path [connector] (first) -- (second);
    \end{tikzpicture}
    \caption{Принцип классификации}
  \end{figure}
\end{frame}


  


\begin{frame}{Двухшаговый процесс классификации}
  \textbf{Классификация} состоит из:
  \begin{enumerate}
    \item Фазы обучения, при котором алгоритм классификации анализирует тестовый набор данных, составленных из пар ``данные — метка класса''. 
      \pause
    \item Этапа классификации по новым данным. Измеряется точность корректно определенных данных к соответсвующему классу.
  \end{enumerate}
\end{frame}

\begin{frame}{Классификация vs. Кластеризация}
  \begin{block}{Замечание}
    \textbf{Классификация} соотносит данные по заранее определенным классам. \textbf{Кластеризация} находит новые зависимости и составляет новые классы, основываясь только на данных.
  \end{block}
\end{frame}

\begin{frame}{Процесс классификации}
  \begin{figure}
    \begin{tikzpicture}
      \node at (0,0) (db1) [db] {Тестовые данные};
      \node at (0,-1.5) (alg) [terminator] {Алгоритм классификации};
      \node at (0,-3.7) (db2) [db] {Правила классификации};
      \path [connector] (db1) -- (alg);
      \path [connector] (alg) -- (db2);
    \end{tikzpicture}
    \caption{Процесс класификации данных: обучение}
  \end{figure}
\end{frame}

\begin{frame}{Процесс классификации}
  \begin{figure}
    \begin{tikzpicture}
      \node at (-2,-4) (db1) [db] {Тестовые данные};
      \node at (0,0) (alg) [terminator] {Правила классификации};
      \node at (2,-4) (db2) [db] {Новые данные};
      \path [2connector] (db1) -- (alg);
      \path [2connector] (alg) -- (db2);
    \end{tikzpicture}
    \caption{Процесс класификации данных: классификация}
  \end{figure}
\end{frame}

\subsection{Примеры}



\begin{frame}{Пример классификации}
  \begin{table}
    \setlength\arrayrulewidth{1pt}
    \rowcolors{1}{gray!20}{white}
    \begin{tabular}{|c c c c|}
            \hline
          \textbf{Имя} & \textbf{Возраст} & \textbf{Доход} & \textbf{Оценка\_кредита} \\ \hline
            Николай & Пожилой  & Средний   & Разрешен   \\ \hline
            Елена & Средний  & Высокий   & Разрешен  \\  \hline
            Костантин & Молодой & Низкий   & Риск   \\\hline
            Олег & Средний & Средний   & Риск   \\\hline
            Артемий & Пожилой  & Низкий   & Разрешен   \\\hline
            Роман & Средний  & Высокий   & Разрешен   \\\hline
            Кристина & Молодой & Низкий   & Риск   \\
            \hline
          \end{tabular}
    \caption{Тестовые данные}
  \end{table}
\end{frame}


\begin{frame}{Пример классификации}
  \begin{table}
    \begin{tabular}{|c|}
      \hline
      IF Возраст = Молодой THEN →\\\
      Оценка\_кредита = Риск \\
      \hline
      IF Доход = Высокий THEN →\\
      Оценка\_кредита = Разрешен \\
      \hline
      IF Возраст = Средний AND Доход = Низкий THEN →\\
      Оценка\_кредита = Риск \\
      \hline
      \dots\\
      \hline
    \end{tabular}
    \caption{Правила классификации}
  \end{table}

\end{frame}


\begin{frame}{Пример классификации}
  \setbeamercovered{}
  Джон, среднего возраста с низкой зарплатой. Какая классификация будет предложена по выделенным правилам?
  \uncover<2->{\textbf{Оценка\_кредита = Риск}}

\end{frame}

\section{Базовые алгоритмы}
\subsection{Деревья решений}

\begin{frame}{Определение дерева решений}
\begin{block}{Определение}
  \textbf{Дерево решений} — это метод классификации, при котором выполняется иерархическое деление данных в виде дерeва, где конечные узлы представляют собой классы данных\footnotemark.
\end{block}\footcitetext{Aggarwal2014}
\end{frame}




\begin{frame}{Пример дерева решений}
\begin{figure}
    \begin{tikzpicture}
      \node [terminator] at (0,4) (t1) {\textit{Возраст?}};
      \node [terminator] at (-3,2) (t2) {\textit{Студент?}};
      \node [terminator] at (3,2) (t8) {\textit{Кредитная история?}};
      \node [decision] at (-4,0) (t3) {Нет};
      \node [decision] at (-2,0) (t4) {Да};
      \node [decision] at (0,2) (t5) {Да};
      \node [decision] at (2,0) (t6) {Нет};
      \node [decision] at (4,0) (t7) {Да};
      \path [connector] (t1) -- (t2) node[midway,above left] {Молодой};
      \path [connector] (t1) -- (t8) node[midway,above right] {Пожилой};
      \path [connector] (t1) -- (t5) node[midway,centered] {Средний};
      \path [connector] (t2) -- (t3) node[midway,left] {Нет};
      \path [connector] (t2) -- (t4) node[midway,right] {Да};
      \path [connector] (t8) -- (t6) node[midway,left] {Посредственная};
      \path [connector] (t8) -- (t7) node[midway,right] {Безупречная};

    \end{tikzpicture}
    \caption{Дерево решений}
  \end{figure}

\end{frame}


\begin{frame}{Определение дерева решений}
\begin{equation} 
  Gini(x_i) = \sum_{y\in Y} {p_{iy}(1-p_{iy})}= 1 - \sum_{y\in Y}{p^{2}_{iy}}
\end{equation}
\end{frame}



\subsection{Байесовы методы классификации}

\begin{frame}{Make Titles Informative.}
\end{frame}

\section{Продвинутые алгоритмы}
\subsection{Метод опорных векторов}

\begin{frame}{Make Titles Informative.}
\end{frame}

\section{Заключение}

\begin{frame}{Заключение}

  % Keep the summary *very short*.
  \begin{itemize}
    \item
      The \alert{first main message} of your talk in one or two lines.
    \item
      The \alert{second main message} of your talk in one or two lines.
    \item
      Perhaps a \alert{third message}, but not more than that.
  \end{itemize}

  % The following outlook is optional.
  \vskip0pt plus.5fill
  \begin{itemize}
    \item
      Outlook
      \begin{itemize}
        \item
          Something you haven't solved.
        \item
          Something else you haven't solved.
      \end{itemize}
  \end{itemize}
\end{frame}

\begin{frame}[allowframebreaks]{Список источников} 
  \nocite{*}
  \printbibliography
\end{frame}

\begin{frame}{Вопросы к аудитории} 

\end{frame}

\begin{frame}
  \titlepage
\end{frame}

\end{document}



